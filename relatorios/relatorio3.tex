\documentclass[letterpaper,twocolumn,10pt]{article}
\usepackage{relatorio,cite}
\usepackage[brazilian]{babel}
\usepackage[utf8]{inputenc}
\usepackage{lmodern}

\begin{document}

\title{
	\large \rm UFSC / CTC / INE\\
	\large \rm Disciplina: Paradigmas de Programação (INE5416)\\
	\Large \bf Relatório 3: Linguagens multiparadigmas
}

\author{
	Professor: Dr. João Dovicchi\\
	\and
	Aluno: Matheus Silva P. Bittencourt\\(15200617)
}

\maketitle

\thispagestyle{empty} % Desativa numeração da página

\section{Linguagens multiparadigmas}

Uma linguagem multiparadigma é um linguagem de programação que suporta
diferentes paradigmas de programação. O objetivo dessas linguagens é
possibilitar ao programador que escolha o melhor `estilo' de programação que se
encaixa na sua tarefa, assumindo que nenhum paradigma resolve todos os
problemas da maneira mais fácil ou da maneira mais eficiente. Uma série de
linguagens se encaixam nesse perfil, por exemplo, F\# e Scala, que implementam
os paradigmas mais comuns(imperativo, orientado a objetos, ...) e dão suporte
completo a programação funcional.

\section{Python}

Python é uma linguagem de alto nível e de propósitos gerais. É uma linguagem
interpretada e que enfatiza a legibilidade do código. Python suporta múltiplos
paradigmas de programação, incluindo orientação a objetos, reflexão, funcional,
entre outros.

Isso possibilita códigos flexíveis e que fazem muito em poucas
linhas, por exemplo, um `Hello World' em Python possui uma linha:
\begin{verbatim}
print("Hello World")
\end{verbatim}
ao mesmo tempo em que o código:
\begin{verbatim}
f = lambda t: "\n".join(map(\
    " ".join, zip(*([iter(t)] * 9))))
\end{verbatim}
define uma função \texttt{f} que tem como parâmetro uma string no formato
\texttt{"72589346184165739239614275847351682916842\
9537952378146234761985687935214519284673"} e transforma em uma matriz formatada
assim:
\begin{verbatim}
7 2 5 8 9 3 4 6 1
8 4 1 6 5 7 3 9 2
3 9 6 1 4 2 7 5 8
4 7 3 5 1 6 8 2 9
1 6 8 4 2 9 5 3 7
9 5 2 3 7 8 1 4 6
2 3 4 7 6 1 9 8 5
6 8 7 9 3 5 2 1 4
5 1 9 2 8 4 6 7 3
\end{verbatim}

Como podemos ver, utilizando os múltiplos paradigmas disponibilizados no Python
podemos combiná-los, para fazer tarefas complexas em algumas linhas, dando
muita robustez a linguagem.

\end{document}
