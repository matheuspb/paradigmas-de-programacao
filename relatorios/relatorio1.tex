\documentclass[letterpaper,twocolumn,10pt]{article}
\usepackage{relatorio,cite}
\usepackage[brazilian]{babel}
\usepackage[utf8]{inputenc}
\usepackage{lmodern}

\begin{document}

\title{
	\large \rm UFSC / CTC / INE\\
	\large \rm Disciplina: Paradigmas de Programação (INE5416)\\
	\Large \bf Relatório 1: Panorama histórico
}

\author{
	Professor: Dr. João Dovicchi\\
	\and
	Aluno: Matheus Silva P. Bittencourt\\(15200617)
}

\maketitle

\thispagestyle{empty} % Desativa numeração da página

\section{Lógica combinatória}

A lógica combinatória, foi introduzida por Moses Schönfinkel~\cite{Schonfinkel}
e Haskell Curry~\cite{Curry}, e é uma notação para eliminar a necessidade de
variáveis na lógica matemática. Recentemente é usada em Ciência da Computação
como um modelo de computação teórico e também como base para projetar
linguagens de programação funcionais. Ela é baseada em combinadores, que são
funções de alta ordem.

\subsection*{Combinadores SKI}

Supondo que \(xy\) seja uma ``função'' \(x\) aplicada a um ``argumento'' \(y\),
podemos descrever a avaliação dos combinadores \textbf{SKI} como:

\begin{enumerate}
	\item \(\mathbf{I}x = x\)
	\item \(\mathbf{K}xy = x\)
	\item \(\mathbf{S}xyz = xz(yz)\)
\end{enumerate}

\subsection*{Combinadores vs. Operadores}

Combinadores são generalizados o suficiente para não dependerem do que seus
termos são, e eles apenas atuam nos seus termos. Operadores são funções, logo
agem num domínio bem definido de possíveis argumentos e possíveis resultados.

\section{Tese de Church-Turing}

Em Teoria da Computação, a tese de Church-Turing é uma hipótese sobre a
natureza das funções computáveis. Ela diz que uma função sobre os números
naturais é computável por um humano seguindo um algoritmo, ignorando as
limitações de recursos, se somente se for computável por uma Máquina de Turing.

Antes da definição precisa de função computável, matemáticos usavam
frequentemente o termo informal ``efetivamente calculável'' para descrever
funções que eram computáveis com métodos que usavam papel e caneta. Já que a
definição de função computável é vaga, a tese não pode ser formalmente provada.

\section{Linguagens de programação e seus paradigmas}

Existem inúmeras linguagens de programação e paradigmas de programação, entre
os mais conhecidos e usados, podemos citar:

\begin{itemize}
	\item Paradigma imperativo (e.g. C, Go, MATLAB)
	\item Paradigma orientado a objetos (e.g. C++, Python, Smalltalk)
	\item Paradigma funcional (e.g. Haskell, Lisp)
	\item Paradigma lógico (e.g. Prolog, SQL, HTML)
\end{itemize}

\section{Lei de Moore, a palestra de Feynman, Peter Shor e a fatoração de
números inteiros grandes}

A lei de Moore é uma observação de que o número de transistores em um circuito
integrado denso, dobra aproximadamente a cada dois anos. No seu artigo de 1965
~\cite{Moore}, Gordon Moore descreveu que o número de componentes por circuito
integrado dobrava a cada ano, e projetou que esse ritmo de crescimento
continuaria por pelo menos mais uma década. Em 1975 ele revisou sua previsão
para a próxima década, falando que dobraria a cada dois anos. Essa previsão se
manteve correta até por volta de 2012.

Richard Feynman, um grande físico do século XX, teve ideias muito à frente de
seu tempo sobre nanotecnologia, e contribuiu conceptualmente, mas diretamente,
para a evolução dessa ramificação da tecnologia, tornando-a muito mais comum
nos tempos atuais.

Peter Shor é conhecido primariamente por formular um algoritmo de fatoração de
números inteiros utilizando computadores quânticos~\cite{Shor}, que funciona
exponencialmente mais rápido do que o melhor algoritmo de fatoração clássico
existente.

\section{Qubit}

Em computação quântica um qubit é uma unidade de informação quântica, análogo
ao bit clássico. Um qubit é um sistema de mecânica quântica de dois estados.
Num sistema clássico, um bit só pode estar em um dos dois estados, no entanto a
mecânica quântica possibilita que o qubit esteja numa superposição de ambos os
estados ao mesmo tempo, uma propriedade fundamental para a computação quântica.

\section{D-Wave}

O D-Wave One é o primeiro computador quântico disponível comercialmente, ele
opera num chip de 128-qubit. Em maio de 2013, uma colaboração entre a NASA,
Google e a Universities Space Research Asscociation (USRA) lançou o Quantum
Artificial Intelligence Lab (Laboratório de Inteligência Artificial Quântica)
baseado no D-Wave Two de 512 qubit que seria usado em pesquisa de aprendizado
de máquina, entre outros campos de estudo.

\footnotesize \bibliographystyle{acm} \bibliography{relatorio1}

\end{document}
