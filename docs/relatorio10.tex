\documentclass[a4paper,twocolumn,10pt]{article}
\usepackage{relatorio,cite,fullpage,lmodern}
\usepackage[brazilian]{babel}
\usepackage[utf8]{inputenc}

\begin{document}

\title{
	\large \rm UFSC / CTC / INE\\
	\large \rm Disciplina: Paradigmas de Programação (INE5416)\\
	\Large \bf Relatório 10: Mônadas
}

\author{
	Professor: Dr. João Dovicchi\\
	\and
	Aluno: Matheus Silva P. Bittencourt\\(15200617)
}

\maketitle

\section{Homologia algébrica}

O conceito de homologia algébrica envolve outras definições abstratas
utilizadas para construir este estudo. Um espaço topológico pode ser definido
como um conjunto de pontos e suas vizinhanças, que satisfazem um conjunto de
axiomas relacionando estes. Permitem formalizar conceitos como convergência e
continuidade. A homologia algébrica é o ramo da matemática que estuda estes
espaços em um ambiente de caráter algébrico.

\section{Functores}

Um functor, em teoria das categorias (ramo da matemática que trata de formas
abstrata estruturas matemáticas), é um mapeamento entre categorias que preserva
estruturas. Pode-se pensar que um functor é um homomorfismo entre categorias.

\section{Mônadas}

Uma mônada (ou tripla, tríade, construção padrão, construção fundamental) é um
\textit{endofunctor}, ou seja, um functor que mapeia uma categoria para ela
mesma, unido a duas transformações naturais. Mônadas são utilizadas na teoria
dos pares de functores adjuntos, e generalizam operações de fechamento em
\textit{posets} para categorias arbitrárias.

\section{Cálculo da resistência de resistores em paralelo}

É possível perceber claramente o uso de uma abordagem monádica no código
produzido em Haskell, embora ainda mostre-se necessário o uso de estruturas
condicionais em uma parte crítica do código. Esquiva-se de uma divisão com zero
verificando se algum dos termos é zero, retornando o tipo \texttt{Nothing}; a
definição explícita de uma função soma segue logo abaixo; e por fim, a
resistência de dois resistores é obtida aplicando, em cadeia, todas as mônadas
implementadas. O tratamento de divisões por zero, neste caso, torna-se muito
mais intuitivo, protegendo o código de tais erros.

\end{document}
