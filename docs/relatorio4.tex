\documentclass[letterpaper,twocolumn,10pt]{article}
\usepackage{relatorio,cite}
\usepackage[brazilian]{babel}
\usepackage[utf8]{inputenc}
\usepackage{lmodern}
\usepackage[colorlinks=true, citecolor=black, urlcolor=blue]{hyperref}
\usepackage{fancyvrb}

\begin{document}

\title{
	\large \rm UFSC / CTC / INE\\
	\large \rm Disciplina: Paradigmas de Programação (INE5416)\\
	\Large \bf Relatório 4: Cálculo-$\lambda$
}

\author{
	Professor: Dr. João Dovicchi\\
	\and
	Aluno: Matheus Silva P. Bittencourt\\(15200617)
}

\maketitle

\thispagestyle{empty} % Desativa numeração da página

\section{Bases do cálculo-$\lambda$}

\textit{An unsolvable problem of elementary number theory}~\cite{Church} pode
ser obtido
\href{http://www.ics.uci.edu/~lopes/teaching/inf212W12/readings/church.pdf}
{aqui}, enquanto existem apenas excertos de \textit{The Calculi of
Lambda-Conversion} na internet.

A principal diferença entre funções matemáticas e expressões lambdas, é que o
domínio de funções é definido por conjunto, ou um produto cartesiano de
conjuntos, já o domínio de expressões lambda é mais amplo, pois pode admitir
inclusive outras expressões lambda. Expressões lambda também só possuem uma
variável, apesar de podermos escrever funções de várias variáveis na forma de
expressões lambda utilizando o conceito de \textit{currying}.

Abaixo segue um pequeno resumo sobre os três tipos de reduções no
cálculo-$\lambda$:

\begin{itemize}

	\item \textbf{conversão-$\alpha$} (ou conversão alfa): responsável por
		renomear variáveis se assim for necessário para o escopo da expressão.
		Por exemplo: $\lambda x.x \stackrel{\alpha}{\rightarrow} \lambda y.y$.

	\item \textbf{redução-$\beta$} (ou redução beta): a mais comum das
		operações de redução por uma grande margem, habilita o processo de
		calcular um resultado da aplicação de uma função a uma expressão. Por
		exemplo: $\lambda x.x + y\ (7) \stackrel{\beta}{\rightarrow} 7 + y$.

	\item \textbf{conversão-$\eta$} (ou conversão eta): elimina redundâncias
		nas abstrações, no caso de uma função ser utilizada apenas para passar
		seu argumento a outras expressões. Por exemplo: $\lambda x.Mx
		\stackrel{\eta}{\rightarrow} M$, onde $x$ não pode ser uma variável
		livre em $M$.

\end{itemize}

\section{Exercícios Haskell}

\begin{enumerate}

	\item
		\begin{Verbatim}[tabsize=0, fontsize=\small]
		deleteBy (\x y -> y `mod` x == 0) 3 [5..10]
		\end{Verbatim}

	\item
		\begin{Verbatim}[tabsize=0, fontsize=\small]
		filter (\x -> x `mod` 4 == 0) [4..19]
		\end{Verbatim}

	\item
		\begin{Verbatim}[tabsize=0, fontsize=\small]
		[1,1,1,2,2,3,3,4]
		\end{Verbatim}

\end{enumerate}

\footnotesize \bibliographystyle{acm}
\bibliography{relatorio4}

\end{document}
