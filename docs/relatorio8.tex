\documentclass[a4paper,twocolumn,10pt]{article}
\usepackage{relatorio,cite,fullpage,lmodern,url}
\usepackage[brazilian]{babel}
\usepackage[utf8]{inputenc}

\begin{document}

\title{
	\large \rm UFSC / CTC / INE\\
	\large \rm Disciplina: Paradigmas de Programação (INE5416)\\
	\Large \bf Relatório 8: Listas e \textit{Arrays}
}

\author{
	Professor: Dr. João Dovicchi\\
	\and
	Aluno: Matheus Silva P. Bittencourt\\(15200617)
}

\maketitle

\thispagestyle{empty} % Desativa numeração da página

\section{\textit{Arrays} em C vs Listas em Python}

A diferença básica dessas estruturas são que os \textit{Arrays} em C são
estruturas estáticas, ou seja, possuem um tamanho fixo (definido em tempo de
compilação) e estão alocadas contiguamente na memória. Para obter um espaço de
memória arbitrário (em tempo de execução), precisamos alocar dinamicamente a
memória usando as funções da família \texttt{malloc}.

Já em Python, a estrutura básica para armazenar dados de forma linear, é a
lista, que é uma estrutura de dados, ou seja, possui operações definidas
(inserção em qualquer posição, remoção em qualquer posição...), que abstraem a
manipulação dos dados. Essa estrutura de dados pode possuir diferentes
implementações, porém normalmente se implementa utilizando-se
\textit{arrays}~\cite{python-lista} por possuírem acesso rápido em qualquer
posição da estrutura. Sabendo disso, espera-se que lidar com estruturas
lineares em Python seja muito menos verboso e trabalhoso do que em C, vide o
código implementado nesse roteiro.

Em C sempre que um \textit{array} é passado por argumento, é passado um
ponteiro, ou seja, se a função modifica o array modificará também o
\textit{array} original. O mesmo acontece em Python com listas, pois como
listas são objetos, então quando passadas como argumento, a função recebe uma
referência para o objeto original. Para que isso não aconteça, é necessário que
se copie as estruturas previamente, antes de passá-las para funções, ou a
própria função pode realizar essa tarefa. Em C podemos utilizar parâmetros
tipados como \texttt{const} para termos certeza que uma função, que não
precise, não modifique o \textit{array}.

\footnotesize \bibliographystyle{acm}
\bibliography{relatorio8}

\end{document}
