\documentclass[a4paper,twocolumn,10pt]{article}
\usepackage{relatorio,cite,fullpage,lmodern}
\usepackage[brazilian]{babel}
\usepackage[utf8]{inputenc}
\usepackage{lmodern}

\begin{document}

\title{
	\large \rm UFSC / CTC / INE\\
	\large \rm Disciplina: Paradigmas de Programação (INE5416)\\
	\Large \bf Relatório 2: Estrutura das linguagens
}

\author{
	Professor: Dr. João Dovicchi\\
	\and
	Aluno: Matheus Silva P. Bittencourt\\(15200617)
}

\maketitle

\thispagestyle{empty} % Desativa numeração da página

\section{Geradores de gramática}

Os geradores de gramática, são, os geradores dos analisadores léxicos e dos
analisadores sintáticos , que podem ser combinados para gerar o \textit{parser}
de uma linguagem.

\subsection*{Analisadores léxicos (Lex, Flex)}

O analisador léxico, também chamado de \textit{tokenizer}, é responsável por
converter uma sequência de caracteres em uma sequência de \textit{tokens}, i.e.
uma sequência de strings com respectivo significado.

Lex é um programa que gera analisadores léxicos. Geralmente é utilizado em
conjunto com o Yacc, entregando ao usuário o código-fonte do analisador na
linguagem de programação C. Recomenda-se produzir a entrada do analisador
sintático com este programa a partir de expressões regulares. Internamente, ele
funciona através de autômatos finitos determinísticos (máquinas que aceitam
linguagens regulares). Flex é a versão aberta do Lex.

\subsection*{Analisadores sintáticos (Yacc, Bison)}

O analisador sintático, ou \textit{parser}, é responsável por construir a
árvore sintática, do programa. Estes programas executam a segunda fase do
funcionamento de um compilador, analisando se a sequência de \textit{tokens}
faz sentido para a sintaxe especificada pela gramática. Por exemplo o código em
python:
\begin{verbatim}
x = 1
print(y)
\end{verbatim}
gera uma árvore sintática correta, porém o erro só é identificado na análise
semântica, pois o analisador sintático não faz a análise sensível a contexto.

Estas ferramentas são essenciais para a formalização da descrição de uma
linguagem de programação e intrínsecas ao funcionamento de computadores
atualmente.

Yacc é um programa que constrói o analisador sintático de uma linguagem,
baseado em uma gramática de livre contexto . Desenvolvido na AT\&T em meados da
década de 1970, existem diversas versões do Yacc, sejam elas abertas ou não.
Bison é a versão aberta do Yacc.

\section{Sintaxe da linguagem C}

Como podemos observar, o arquivo \texttt{scan.l}, é usado para gerar o
analisador léxico, pois nele estão definidas todas as palavras reservadas, e
algumas expressões regulares, que são usadas para reconhecer e definir os
\textit{tokens} que serão passados ao analisador sintático. Por exemplo,
algumas expressões regulares são usadas para identificar os \textit{tokens}
\texttt{CONSTANT} , alguns símbolos especiais (e.g. `+', `-', `;') possuem
\textit{tokens} semelhantes ao próprio símbolo.

O arquivo \texttt{gram.y} é usado para gerar o analisador sintático. Como
podemos ver ele possui uma série de `regras' que definem como os
\textit{tokens} devem estar ordenados para que o programa tenha validade
sintática. Por exemplo, um \texttt{if} precisa estar no formato \texttt{if
(expr)}, logo se num programa existir um \texttt{if ()} ele não compilará, pois
o analisador sintático detectará um erro, já que não existe uma expressão
válida entre os \textit{tokens} \texttt{(} e \texttt{)}.

\end{document}
