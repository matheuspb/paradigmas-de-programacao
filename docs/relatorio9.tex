\documentclass[a4paper,twocolumn,10pt]{article}
\usepackage{relatorio,cite,fullpage,lmodern}
\usepackage[brazilian]{babel}
\usepackage[utf8]{inputenc}

\begin{document}

\title{
	\large \rm UFSC / CTC / INE\\
	\large \rm Disciplina: Paradigmas de Programação (INE5416)\\
	\Large \bf Relatório 9: Listas em Haskell
}

\author{
	Professor: Dr. João Dovicchi\\
	\and
	Aluno: Matheus Silva P. Bittencourt\\(15200617)
}

\maketitle

\thispagestyle{empty} % Desativa numeração da página

\section*{Parte 1}

\subsection*{\textit{Lazyness}}

\textit{Lazyness evaluation} ou avaliação preguiçosa, é uma estratégia para
avaliar expressões de tal forma que esta é atrasada até que o valor da
expressão seja realmente necessário. Isso possibilita que se crie estruturas de
dados virtualmente infinitas, pois os valores dela só serão calculados ao passo
em que são requisitados.

Normalmente essa técnica é combinada com outra, chamada \textit{memoization},
que envolve guardar os valores de uma função que já foram computados, numa
tabela, de tal forma que na próxima vez que a função for chamada, com os mesmos
parâmetros, o valor já é conhecido. Isto é possível em linguagens puramente
funcionais, como Haskell, pois todas as funções são puras, logo, o valor de uma
função é sempre o mesmo quando chamada com os mesmos parâmetros.

\subsection*{\texttt{map}}

Em Haskell, podemos escrever funções de alta ordem, ou seja, funções que
recebem outras funções como parâmetro. Uma das mais usadas é a função
\texttt{map}, que recebe uma função e uma lista como parâmetros e retorna outra
lista, o que ela faz é para cada elemento da lista, executar a função dada e
colocar o resultado na lista de retorno, ou seja, ela mapeia os valores da
lista para uma nova lista de acordo com a função passada. Exemplo com o
\texttt{ghci}:

\begin{verbatim}
Prelude> map (^ 2) [1..5]
[1,4,9,16,25]
\end{verbatim}

Neste caso, o \texttt{map} eleva cada elemento da lista de 1 a 5 ao quadrado.

\subsection*{\texttt{Data.List}}

O módulo \texttt{Data.List} é composto de diversos utilitários para manipulação
de listas, como um operador para concatenação destas, testes para lista vazia,
retorno de tamanho de lista, funções para reversão, mapeamento, retorno de
subsequências e permutações, entre outros.

\section*{Parte 2}

As implementações podem ser encontradas no arquivo \verb|pa.hs|, para o cálculo
da função $\Gamma$ foi utilizado uma aproximação de Stirling~\cite{Nemes2010}:

\[
\Gamma(z) \approx \sqrt{\frac{2\pi}{z}}
\left( \frac{1}{e} \left( z + \frac{1}{12z - \frac{1}{10z}} \right) \right) ^z
\]

\footnotesize \bibliographystyle{acm}
\bibliography{relatorio9}

\end{document}
