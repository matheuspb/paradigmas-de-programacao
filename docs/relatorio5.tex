\documentclass[letterpaper,twocolumn,10pt]{article}
\usepackage{relatorio,cite}
\usepackage[brazilian]{babel}
\usepackage[utf8]{inputenc}
\usepackage{lmodern}

\begin{document}

\title{
	\large \rm UFSC / CTC / INE\\
	\large \rm Disciplina: Paradigmas de Programação (INE5416)\\
	\Large \bf Relatório 5: Análise Léxica: Sintaxe do Haskell
}

\author{
	Professor: Dr. João Dovicchi\\
	\and
	Aluno: Matheus Silva P. Bittencourt\\(15200617)
}

\maketitle

\thispagestyle{empty} % Desativa numeração da página

\section*{Parte 1}

\begin{verbatim}
f x = case x of
	0 -> 1
	1 -> 5
	2 -> 2
	_ -> 1

quicksort :: (Ord a) => [a] -> [a]
quicksort [] = []
quicksort (x:xs) = quicksort lt ++ [x] ++ quicksort ge
	where {lt = [y | y <- xs, y < x]; ge = [y | y <- xs, y >= x]}
\end{verbatim}

\section*{Parte 2}

\begin{enumerate}
	\item \verb$[1..1000]$
	\item \verb$[1,4..99]$
	\item \verb$pg a1 = [a1*(2**(x-1)) | x <- [1..50]]$\\
		Sendo que, a1 é o primeiro termo da progressão geométrica.
	\item
		\begin{verbatim}
			-- lista dos infinitos fatoriais
			fatoriais = 1 : zipWith (*) fatoriais [1..]

			-- enésimo termo da lista infinita
			fat n = fatoriais !! n
		\end{verbatim}
\end{enumerate}

\end{document}
