\documentclass[a4paper,twocolumn,10pt]{article}
\usepackage{relatorio,cite,fullpage,lmodern,amsmath}
\usepackage[brazilian]{babel}
\usepackage[utf8]{inputenc}

\begin{document}

\title{
	\large \rm UFSC / CTC / INE\\
	\large \rm Disciplina: Paradigmas de Programação (INE5416)\\
	\Large \bf Relatório 7: Módulos em Haskell
}

\author{
	Professor: Dr. João Dovicchi\\
	\and
	Aluno: Matheus Silva P. Bittencourt\\(15200617)
}

\maketitle

\thispagestyle{empty} % Desativa numeração da página

\section*{Módulo \texttt{Hyperbolic}}

No arquivo \texttt{hyperbolic.hs} é definido o módulo \texttt{Hyperbolic}, que
possui as funções hiperbólicas ($sinh$, $cosh$, $tanh$, $coth$). Para calcular
essas funções, precisamos do valor de $e$ (base do logaritmo natural).  Sendo
$e = \sum_{n=0}^{\infty} \frac{1}{n!}$, pegando apenas os primeiros $20$ termos
dessa série, já obtemos o valor de $e$ com uma boa precisão (valor igual a
função \texttt{exp 1} do módulo \texttt{Prelude}).

Agora basta definir as funções hiperbólicas:

\begin{align*}
	sinh(u) &= \frac{e^u - e ^{-u}}{2} \\
	cosh(u) &= \frac{e^u + e ^{-u}}{2} \\
	tanh(u) &= \frac{sinh(u)}{cosh(u)} \\
	coth(u) &= \frac{cosh(u)}{sinh(u)}
\end{align*}

Para que não haja ambiguidade com as funções hiperbólicas do módulo
\texttt{Prelude}, basta adicionar a linha \texttt{import Prelude hiding (sinh,
cosh, tanh, coth)}

Alguns testes feitos no \texttt{ghci}:

\begin{verbatim}
*Hyperbolic> import Prelude

*Hyperbolic Prelude> Prelude.exp 1
2.718281828459045
*Hyperbolic Prelude> Hyperbolic.e
2.718281828459045

*Hyperbolic Prelude> Prelude.sinh 1.5
2.1292794550948173
*Hyperbolic Prelude> Hyperbolic.sinh 1.5
2.1292794550948173

*Hyperbolic Prelude> Prelude.cosh 1.2
1.8106555673243747
*Hyperbolic Prelude> Hyperbolic.cosh 1.2
1.8106555673243747
\end{verbatim}

\end{document}
